\documentclass[paper=a4,fontsize=12pt,ngerman]{scrartcl}

\usepackage{float}
\usepackage{tikz}
\usepackage{pgf-pie}
\usepackage{caption}
\usepackage{longtable} % für lange Tabellen
\usepackage{fancyhdr}
\usepackage{hyperref}
\usepackage{lipsum} % für Blindtext
\usepackage{pgfplots}
\pgfplotsset{compat=1.18} % Setze die Kompatibilität
\usepackage{pgfplotstable}
\usepackage[utf8]{inputenc}				
\usepackage[T1]{fontenc}
\usepackage{graphicx}
\usepackage[ngerman]{babel}
\usepackage{amsmath}
\usepackage[a4paper,left=25mm,right=35mm,top=25mm,bottom=30mm]{geometry}
\usepackage{parskip}
\usepackage{natbib}
\usepackage{url}
\usepackage{booktabs}
\usepackage{siunitx}
\sisetup{output-decimal-marker={,}}

\begin{document}

\pagenumbering{roman}
\pagestyle{plain}

% Einbinden der Titelseite
\begin{titlepage}

\linespread{1.5}

\includegraphics[width=\linewidth]{graphics/htw_logo}

\begin{center}
    \large  
    \hfill
    \vfill
    \Large{\bfseries{Vibe Coding}}
    
    von \\
     Tony Adam S Nsangou Tatsinkou

    \vfill
		
    Ein wissenschaftlicher Bericht im Rahmen der Vorlesung\\
    \glqq Wissenschaftliches Arbeiten\grqq\\
    an der htw saar im Studiengang Informatik\\
	
    \vfill	
    \vfill
	
    Saarbrücken, den 17.07.2020
\end{center}
    
\end{titlepage}

% Hier ist der Abstract
\section*{Abstract}
Diese Arbeit beschäftigt sich mit dem Konzept des Vibe Coding, einer modernen Form der Softwareentwicklung, bei der Künstliche Intelligenz (KI) eine zentrale Rolle spielt. Ziel ist es, die Potenziale und Herausforderungen von KI-gestützten Programmierwerkzeugen zu analysieren und deren Auswirkungen auf die Qualität von Software sowie auf die Weiterentwicklung von Entwicklerinnen und Entwicklern zu bewerten. Es wird untersucht, inwiefern Vibe Coding den Zugang zur Programmierung erleichtert, aber auch welche Risiken durch einen möglichen Kompetenzverlust entstehen können. Abschließend werden Lösungsansätze vorgestellt, um einen verantwortungsvollen und nachhaltigen Einsatz von KI in der Softwareentwicklung zu fördern.

\clearpage
\section*{Selbstständigkeitserklärung}
Ich versichere, dass ich die vorliegende Arbeit selbstständig verfasst und 
keine anderen als die angegebenen Quellen und Hilfsmittel benutzt habe.
Insbesondere habe ich alle KI-basierten Werkzeuge angegeben, die ich bei
der Erstellung, Übersetzung oder Überarbeitung des Textes verwendet habe.

Ich erkläre hiermit weiterhin, dass die vorgelegte Arbeit zuvor weder von mir 
noch von einer anderen Person an dieser oder einer anderen Hochschule 
eingereicht wurde.

Darüber hinaus ist mir bekannt, dass die Unrichtigkeit dieser Erklärung eine 
Benotung der Arbeit mit der Note \glqq nicht ausreichend\grqq{} zur Folge hat 
und einen Ausschluss von der Erbringung weiterer Prüfungsleistungen zur Folge 
haben kann.
\bigskip
Saarbrücken, den 21.08.2025

\smallskip
\vspace{1cm}
\noindent

\begin{flushright}
    \includegraphics[width=4cm]{draw-signature.png}
\end{flushright}
Unterschrift

% Das Inhaltsverzeichnis
\clearpage
\tableofcontents
\thispagestyle{plain}

\clearpage
\section{Einleitung}
In den letzten Jahren hat sich der Bereich der Programmierung rasant weiterentwickelt.
Ein aktueller Trend ist das sogenannte "Vibe Coding", das zunehmend an Bedeutung gewinnt.
Lösungsansätze zur Bewältigung der damit verbundenen Probleme aufzuzeigen.

\subsection{Grundlagen und Begriffe}
Die Grundlage des Vibe Coding besteht darin, komplexe Aufgaben schneller zu lösen. Die Idee des Vibe Coding wurde von dem slowakisch-kanadischen Informatiker Andrej Karpathy in einem Tweet geprägt. 
Er beschrieb es als eine neue Art zu programmieren, die darauf abzielt, komplexe Aufgaben schneller zu lösen:
\begin{quote}
\textit{There's a new kind of coding I call "Vibe Coding", where you fully give in to the vibes, embrace exponentials, and forget that the code even exists.} 
\citep{TweetAndrejKarpathy}
\end{quote}



\subsection{Definitionen}
Vibe ist ein englischer Begriff, der verwendet wird, um eine bestimmte Stimmung oder Atmosphäre zu beschreiben.
In der Programmierung bezieht sich Vibe auf den Stil und die Qualität des Codes.
Code ist eine Sammlung von Anweisungen, die ein Computer ausführen kann, um eine bestimmte Aufgabe zu erledigen.
Zusammen ergibt sich aus Vibe und Code das sogenannte Vibe Coding, das ermöglicht,
ohne Programmiererfahrung Apps, Websites und andere Software zu erstellen – durch die Verwendung von KI-Tools. 
Ein besseres Verständnis darunter wäre:
Vibe Coding ist eine Art der Softwareentwicklung, bei der man einer KI in einfacher Sprache sagt, was man will, und die KI schreibt den Code für dich. \citep{Datacamp}

\clearpage
\section{Analyse und Bewertung}
In diesem Kapitel werden die aktuellen Entwicklungen und Herausforderungen rund um das Vibe Coding analysiert und bewertet. Zunächst erfolgt eine Betrachtung der aktuellen Situation, anschließend werden mögliche Lösungsansätze vorgestellt.

\subsection{Analyse der aktuellen Situation}
Die aktuelle Situation besagt, dass die Einführung von KI in verschiedene Bereiche der Gesellschaft, 
beispielsweise in der Schule, auch die Einführung von KI in der Softwareentwicklung zeigt. 
Dabei handelt es sich um eine Art von Softwareentwicklung, bei der die KI den Code für dich schreibt,
ohne jegliche Erfahrung in der Softwareentwicklung. Ich habe aus einer Umfrage folgendes Ergebnis erhalten: 

\begin{table}[ht]
    \centering
    \caption{Hauptsächliche Verwendungszwecke von KI-Tools}
    \begin{tabular}{l S[table-format=2.1]}
    \toprule
    \textbf{Kategorie} & \textbf{Anteil (\%)} \\
    \midrule
    Hilfe beim Programmieren (Code schreiben) & 45.3 \\
    Code erklären lassen & 22.7 \\
    Fehler finden und beheben (Debugging) & 15.9 \\
    Hilfe bei Hausaufgaben oder Uni-Projekten & 16.1 \\
    Kommentare oder Dokumentation erstellen & 6.8 \\
    \bottomrule
    \end{tabular}
\end{table}

\begin{figure}[h]
\centering
\begin{tikzpicture}
\begin{axis}[
    ybar stacked,
    bar width=30pt,
    width=\textwidth,
    height=8cm,
    ymin=0,
    ymax=100,
    ylabel={Prozent (\%)},
    symbolic x coords={
        Hilfreich beim Programmieren,
        Verstehen von KI-Code,
        Datenschutz/Plagiate
    },
    xtick=data,
    x tick label style={rotate=15, anchor=east},
    legend style={
        at={(0.5,-0.25)},
        anchor=north,
        legend columns=2
    },
    nodes near coords,
    every node near coord/.append style={font=\scriptsize},
]

% Frage 1: Hilfreich beim Programmieren
\addplot coordinates {(Hilfreich beim Programmieren,65)}; % Sehr hilfreich
\addplot coordinates {(Hilfreich beim Programmieren,35)}; % Ganz okay

% Frage 2: Verstehen von KI-Code
\addplot coordinates {(Verstehen von KI-Code,75)}; % Sehr wichtig
\addplot coordinates {(Verstehen von KI-Code,25)}; % Wichtig

% Frage 3: Datenschutz/Plagiate
\addplot coordinates {(Datenschutz/Plagiate,35)}; % Ja
\addplot coordinates {(Datenschutz/Plagiate,35)}; % Ein bisschen
\addplot coordinates {(Datenschutz/Plagiate,20)}; % Nein
\addplot coordinates {(Datenschutz/Plagiate,10)}; % Noch nie überlegt

\legend{
    Sehr hilfreich / Sehr wichtig / Ja,
    Ganz okay / Wichtig / Ein bisschen,
    Nein,
    Noch nie überlegt
}
\end{axis}
\end{tikzpicture}
\caption{Zusammenfassung der Umfrageergebnisse zu KI-Tools beim Programmieren}
\end{figure}



\begin{table}[h]
\small % Kleinere Schriftgröße
\caption{Meinungen zu Vibe Coding} \label{tab:meinungen}
\begin{tabularx}{\textwidth}{X} % X für automatische Spaltenbreite
    \toprule
    \textbf{Meinung zu Vibe Coding} \\
    \midrule
    Generierter Code oft fehlerhaft, erfordert logisches Verständnis zur Korrektur. \\
    \midrule
    Cool, aber KI-Nutzung nicht immer wünschenswert. \\
    \midrule
    Erhöht Technical Debt erheblich. \\
    \midrule
    Kann nützlich sein, aber Risiken wie Bugs und Inkonsistenzen bestehen. Für Prototypen geeignet, aber in Produktion problematisch. \\
    \midrule
    Schnelle Ergebnisse, aber spätere Weiterarbeit schwierig. Für Prototypen akzeptabel. \\
    \midrule
    Kopieren von KI-Code ohne Verständnis ist problematisch. \\
    \midrule
    „Vibe Coder“ als Begriff wirkt fast beleidigend. \\
    \midrule
    Langfristiges Coding ohne Vorkenntnisse mit KI nicht realistisch. \\
    \bottomrule
\end{tabularx}
\end{table}

\begin{figure}[ht]
    \centering
    \begin{minipage}[t]{0.45\textwidth}
        \centering
        \begin{tikzpicture}
            \pie[pos={8,0}, explode=0.1]{70/Ja, 30/Nein}
        \end{tikzpicture}
        \caption{Denkst du, dass KI-Tools dich beim Programmieren schneller oder produktiver machen?}
    \end{minipage}
    \hfill
    \begin{minipage}[t]{0.45\textwidth}
        \centering
        \begin{tikzpicture}
            \pie[pos={8,0}, explode=0.1]{25/Ja, 10/Nein, 65/Ein bisschen}
        \end{tikzpicture}
        \caption{Haben Lehrende (z.\,B. Professor/innen oder Tutor/innen) erklärt, wie man KI-Tools sinnvoll oder ethisch nutzt?}
    \end{minipage}
\end{figure}
\clearpage

\begin{quote}
A tweet from Peter Yang perfectly captures what I’ve been observing in the field: \\
\textit{Honest reflections from coding with AI so far as a non-engineer: It can get you 70\% of the way there, but that last 30\% is frustrating. It keeps taking one step forward and two steps backward with new bugs, issues, etc. If I knew how the code worked I could probably fix it myself. But since I don’t, I question if I’m actually learning that much.}
\citep{VibeCodingTheFutureOfProgramming}
\end{quote}

\subsection{Lösungsansätze}
Nach der Analyse der Umfrage wo sich leute geäußert haben über Vibe Coding daraus konnten wir einige Lösungsansätze ableiten, um die Herausforderungen des Vibe Codings zu bewältigen und die Qualität der Softwareentwicklung zu sichern:
\begin{itemize}
    \item Förderung von Grundkenntnissen in der Programmierung, auch beim Einsatz von KI-Tools.
    \item Entwicklung von Leitfäden für den sinnvollen Einsatz von KI in der Softwareentwicklung.
    \item Integration von KI-gestütztem Coding als Ergänzung, nicht als Ersatz für klassische Lernmethoden.
    \item Ein Rückgriff auf die archaische Programmierung bedeutet in diesem Fall, dass man bevorzugt Dokumentation liest und viele Quellen wie beispielsweise YouTube, StackOverflow usw. nutzt.
    \item Es soll ein Modul eingefügt werden, in dem Studierende über die Nutzung von KI beraten werden. Außerdem wird dort gezeigt, wie KI genutzt werden kann. Das Modul trägt zur Schaffung einer positiven Arbeitsatmosphäre bei.
    \item Aus beruflicher Perspektive sollte in Unternehmen von einer sehr cleveren Nutzung von KI abgeraten werden um das Vibe Coding zu vermeinden.  
\end{itemize}

\clearpage
\section{Ergebnisse und Diskussion}
In diesem Kapitel werden die wichtigsten Ergebnisse der Analyse zusammengefasst und diskutiert. Es wird aufgezeigt, welche Chancen und Risiken sich durch den Einsatz von Vibe Coding ergeben.

\subsection{Ergebnisse der Analyse}
Die Analyse zeigt, dass Vibe Coding den Zugang zur Softwareentwicklung erleichtert, aber auch die Gefahr birgt, dass grundlegende Kompetenzen verloren gehen.
Außerdem, wenn man das betrachtet, tötet das Grundkonzept der Softwareentwicklung, da sie den Fokus auf das Ergebnis legen und nicht auf den Prozess des Lernens und Verstehens.
Die Ergebnisse der Analyse verdeutlichen, dass Vibe Coding sowohl Chancen als auch Herausforderungen mit sich bringt.
Die Chancen liegen in der schnelleren Umsetzung von Ideen und der Erleichterung des Zugangs zur Programmierung für Nicht-Programmierer.
Die Herausforderungen bestehen darin, dass die Qualität des Codes und das Verständnis für die zugrunde liegenden Konzepte leiden können, wenn KI-Tools ohne ausreichende Kenntnisse und Erfahrung eingesetzt werden.

Es ist wichtig, einen Mittelweg zu finden: KI-Tools können unterstützen, sollten aber nicht das Verständnis für grundlegende Konzepte ersetzen.
Wenn wir von Softwareentwicklung sprechen, ist es wichtig zu verstehen, dass Softwareentwicklung mit geistigen Konzepten verbunden ist wie Geduld, Lesen usw.

\clearpage
\section{Fazit und Schlussfolgerungen}
Vibe Coder fühlt sich als Softwareentwickler, aber wird nicht automatisch einer. 
Vibe Coding kann gut sein, weil man schnellere Lösungen bekommt, aber das effiziente und gute Lösungen zu bekommen, ist durch den wahren Einsatz der Softwareentwicklung möglich – wo man Dokumentation liest, Geduld hat, Fehler macht und daraus lernt, mit Menschen interagiert und in verschiedenen Quellen nach Lösungen sucht.

\subsection{Offene Fragen}
Wird in Zukunft so sein, dass wir mehr im Bereich der Softwareentwicklung auf KI-Tools angewiesen sein werden?  
Wird dadurch die Weiterentwicklung von Menschen, die im Bereich der Softwareentwicklung arbeiten, beeinträchtigt?  
Wie wird die Qualität von Software in Zukunft aussehen, wenn wir uns zu sehr auf KI-Tools verlassen?

\subsection{Diskussion}
Wenn ich als Bachelor-Student der Informatik daran denke, stelle ich mir eine Menge Defizite durch das Vibe Coding vor, bei dem man sich komplett auf KI-Tools verlässt, ohne selbst zu denken.
Ich habe zwar Lösungsansätze vorgestellt, aber in Rahmen dieser Arbeit Konnte ich ein Aspekt ein geeignet methode zu finden der besser als das 
Vibe Coding ist. Dies wäre ein Pair Programming mit einem KI-Tool. Diese Ansatz wurde mir vorgestellt in das Buch 
Vibe Coding The Future Of Programming von Addy Osmani 
\begin{quote}
\textit{Traditional pair programming involves two humans collaborating at one workstation. With the Advent of AI, a hybrid approach has 
emerged : one human developer working alongside an AI assistant. This setup can be particularly effective, offering a blend of human intuition and machine efficiency.}
\end{quote}
\citep{Vibe_Coding_The_Future_Of_Programming2}
Nun möchte zwei diagramm darstellen, die die den Zukunft von Softwareentwicklung widerspiegel anhand von diese zwei Methoden, also 
Pair Programming mit einem KI-Tool vs Vibe Coding.
\begin{figure}[H]
    \centering
    \begin{minipage}[t]{0.45\textwidth}
        \centering
        \begin{tikzpicture}
            \begin{axis}[
                ybar,
                bar width=30pt,
                width=\textwidth,
                height=8cm,
                ymin=0,
                ymax=100,
                ylabel={Prozent (\%)},
                symbolic x coords={Pair Programming mit KI-Tool, Vibe Coding},
                xtick=data,
                x tick label style={rotate=15, anchor=east},
                nodes near coords,
                every node near coord/.append style={font=\scriptsize},
            ]
            \addplot coordinates {(Pair Programming mit KI-Tool,85)};
            \addplot coordinates {(Vibe Coding,40)};
            \end{axis}
        \end{tikzpicture}
        \caption{Zukunft der Softwareentwicklung: Pair Programming mit KI-Tool vs Vibe Coding}
    \end{minipage}
    \hfill
    \begin{minipage}[t]{0.45\textwidth}
        \centering
        \begin{tikzpicture}
            \begin{axis}[
                ybar,
                bar width=30pt,
                width=\textwidth,
                height=8cm,
                ymin=0,
                ymax=100,
                ylabel={Prozent (\%)},
                symbolic x coords={Pair Programming mit KI-Tool, Vibe Coding},
                xtick=data,
                x tick label style={rotate=15, anchor=east},
                nodes near coords,
                every node near coord/.append style={font=\scriptsize},
            ]
            \addplot coordinates {(Pair Programming mit KI-Tool,90)};
            \addplot coordinates {(Vibe Coding,30)};
            \end{axis}
        \end{tikzpicture}
        \caption{Zukunft der Softwareentwicklung: Pair Programming mit KI-Tool vs Vibe Coding (Qualität)}
    \end{minipage}
\end{figure}

\renewcommand\refname{Literaturverzeichnis}
\bibliographystyle{alpha}
\bibliography{references}
\addcontentsline{toc}{section}{Literaturverzeichnis}

% Anhang
\clearpage
\appendix
\part*{Anhang}
\addcontentsline{toc}{section}{Anhang}

\section{Datenmaterial}
Für die Analyse und Bewertung von Vibe Coding wurden verschiedene Quellen herangezogen. Eine zentrale Grundlage bildet das Buch \textit{Vibe Coding: The Future of Programming} von Addy Osmani \citep{VibeCodingTheFutureOfProgramming}. In diesem Werk werden aktuelle Entwicklungen, Anwendungsbeispiele und Herausforderungen im Bereich KI-gestützter Programmierung umfassend dargestellt.

Zusätzlich wurden weitere wissenschaftliche Artikel, Blogbeiträge und Erfahrungsberichte ausgewertet, um ein möglichst breites und aktuelles Datenmaterial zu erhalten. Die Kombination dieser Quellen ermöglicht eine fundierte Analyse der Potenziale und Grenzen von Vibe Coding.

\end{document}
