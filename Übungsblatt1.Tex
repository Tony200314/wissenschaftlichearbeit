\documentclass[paper=a4,fontsize=12pt,ngerman]{scrartcl}

\usepackage[utf8]{inputenc}				
\usepackage[T1]{fontenc}
\usepackage{graphicx}
\usepackage[ngerman]{babel}
\usepackage{amsmath}
\usepackage[a4paper,left=25mm,right=35mm,top=25mm,bottom=30mm]{geometry}
\usepackage{parskip}
\usepackage{natbib}
\usepackage{url}
	

\begin{document}

\pagenumbering{roman}
\pagestyle{plain}

% Einbinden der Titelseite
\begin{titlepage}

\linespread{1.5}

\includegraphics[width=\linewidth]{graphics/htw_logo}

\begin{center}
    \large  
    \hfill
    \vfill
    \Large{\bfseries{Vibe Coding}}
    
    von \\
     Tony Adam S Nsangou Tatsinkou

    \vfill
		
    Ein wissenschaftlicher Bericht im Rahmen der Vorlesung\\
    \glqq Wissenschaftliches Arbeiten\grqq\\
    an der htw saar im Studiengang Informatik\\
	
    \vfill	
    \vfill
	
    Saarbrücken, den 17.07.2020
\end{center}
    
\end{titlepage}

% Hier ist der Abstract
\section*{Abstract}
 

\newpage
\section*{Selbstständigkeitserklärung}
Ich versichere, dass ich die vorliegende Arbeit selbstständig verfasst und 
keine anderen als die angegebenen Quellen und Hilfsmittel benutzt habe.
Insbesondere habe ich alle KI-basierten Werkzeuge angegeben, die ich bei
der Erstellung, Übersetzung oder Überarbeitung des Textes verwendet habe.

Ich erkläre hiermit weiterhin, dass die vorgelegte Arbeit zuvor weder von mir 
noch von einer anderen Person an dieser oder einer anderen Hochschule 
eingereicht wurde.

Darüber hinaus ist mir bekannt, dass die Unrichtigkeit dieser Erklärung eine 
Benotung der Arbeit mit der Note \glqq nicht ausreichend\grqq \ zur Folge hat 
und einen Ausschluss von der Erbringung weiterer Prüfungsleistungen zur Folge 
haben kann.
\bigskip
 
Saarbrücken, den 01.01.1970

\smallskip
Unterschrift




% Das Inhaltsverzeichnis
\clearpage
\tableofcontents 
\newpage

\section{Einleitung}
\subsection{Grundlagen und Begriffe}
\subsection{Definitionen}

\section{Analyse und Bewertung}
\subsection{Analyse der aktuellen Situation}
\subsection{Lösungsansätze}

\section{Ergebnisse und Diskussion}
\subsection{Ergebnisse der Analyse}
\subsection{Diskussion der Ergebnisse}

\section{Fazit und Schlussfolgerungen}

\clearpage
\pagenumbering{arabic}

% Hier beginnt das erste Kapitel
\section{Einleitung}
 In den letzten Jahren hat sich der Bereich der Programmierung rasant weiterentwickelt.
 Ein aktueller Trend ist das sogenannte "Vibe Coding", das zunehmend an Bedeutung gewinnt.
 Mit dieser Entwicklung sind zahlreiche Fragen und Herausforderungen verbunden, 
 die sowohl in der Praxis als auch in der Forschung diskutiert werden.

 Ziel dieser wissenschaftlichen Arbeit ist es, die wichtigsten Fragestellungen
 rund um das Thema Vibe Coding zu analysieren und verschiedene 
 Lösungsansätze zur Bewältigung der damit verbundenen Probleme aufzuzeigen.

\subsection{Grundlagen und Begriffe}
 Die Grundlage des Vibe Coding besteht darin,
 mithilfe eines Tools für Künstliche Intelligenz, 
 wie beispielsweise eines KI-basierten Code-Generators, 
 Programmieraufgaben durch sogenannte Prompts zu steuern. 
 Dabei formuliert der Nutzer Anweisungen in natürlicher Sprache, 
 die von der KI in Programmcode umgesetzt werden.
 Die Vorteile dieser Methode liegen in der Vereinfachung des Programmierprozesses
 und der Möglichkeit, komplexe Aufgaben schneller zu lösen.

\subsection{Definitionen}
Vibe ist ein Englische Begriffe, der verwendet wird, um eine bestimmte Stimmung oder Atmosphäre zu beschreiben.
In der Programmierung bezieht sich Vibe auf den Stil und die Qualität des Codes.
Code ist ein Sammlung von anweisungen, die ein Computer ausführen kann, um eine bestimmte Aufgabe zu erledigen.
Zusammen ergibt sich aus Vibe und Code ein das sogenannte Vibe Coding, das ermöglicht,
ohne programmiererfahrung App, websites und andere Software zu erstellen durch die Verwendung von KI-Tools. 
ein bessere verständnis unter diese wäre :
\begin{document}
 Vibe Coding ist eine Art der Softwareentwicklung,bei der [...] einer KI in einfacher Sprache sagt, was du willst, und die KI schreibt den Code für dich. \citep{Datacamp}
 \bibliographystyle{plainnat}
 \bibliography{references}
\end{document}


\section{Analyse und Bewertung}
\subsection{Analyse der aktuellen Situation}
Die Aktuelle Situation besage dass die einführung von KI in verschiedene Bereiche der Gesellschaft, 
beispielsweise in der Schule zeig auch die einführung von KI in der Softwareentwicklung, aber in 
so einen Art von Softwareentwicklung, dass die KI den Code für dich schreibt,
ohne jegliche erfahrung in der Softwareentwicklung.
\begin{quote}
 A tweet from Peter Yang perfectly captures what I’ve been observing in the field: \\
\textit{Honest reflections from coding with AI so far as a non-engineer: It can get you 70\% of the way there, but that last 30\% is frustrating. It keeps taking one step forward and two steps backward with new bugs, issues, etc. If I knew how the code worked I could probably fix it myself. But since I don’t, I question if I’m actually learning that much.}
\citep{VibeCodingTheFutureOfProgramming}
\bibliography{references}
\end{quote}

\subsection{Lösungsansätze}
genauer zu sein nach der Analyse der Aktuelle Situation, Die Analyse 
zeigt, dass Vibe Coding eine vielversprechende Methode ist, aber sehr große Risiko 
mit sich bringt , da es für Menschen ohne Programmiererfahrung schwierig sein kann. 
Als Lösungsansätze können wir die folgenden Punkte nennen:
   \begin{itemize}
     \item Eine Rückkehr zu Altümliche Methode in der Softwareentwicklung
     \item Das verzichten auf KI Im Studium zu den ersten Semestern 
     \item das Wertschätzen auf Dokumentation wie Oracle, Stackoverflow und GitHub
     \item Das Verzichten auf KI-Tools in Kleinen projekten 
    \end{itemize}


\section{Ergebnisse und Diskussion}
\subsection{Ergebnisse der Analyse} 
Nach unsere Analyse wir können daraus schließen, dass Vibe Coding sehr schlecht Verfahren 
für die Softwareentwicklung ,weil dass die Kreativität der Menschen einschränkt, Die Lernziele 
im Bereich von Studium nicht erreicht werden können. Außerdem reden wir von 70\% das Verzichten auf 
Developper und 30\% das Verzichten auf KI-Tools, also der Fokus hier ist mehr  auf KI-tools als auf der
Menschen, was uns die Frage stellt, ob wir in der Zukunft noch Menschen in Bereich der Softwareentwicklung
brauchen werden. 

\subsection{Diskussion der Ergebnisse}
Die Ergebnisse der Analyse zeigen, dass Vibe Coding zwar eine innovative Methode ist,
aber auch erhebliche Herausforderungen mit sich bringt. für uns Fokus auf Menschen in 
Bereich der Softwareentwicklung ist sehr wichtig, da es die Kreativität und Problemlösungsfähigkeiten fördert.
Außerdem ist es wichtig, dass Studierende die Grundlagen der Programmierung verstehen, bevor sie auf KI-Tools zurückgreifen.
Meine eigene erfahrung zeigt dass mit programmiererfahrung kann man verschiedene Programmiersprachen lernen aber 
die KI-tools sind nur da um uns den weg zu zeigen wohin wir gehen sollen, aber nicht um uns den Weg zu nehmen.
\begin{figure}[h]
\centering
\includegraphics[width=0.5\textwidth]{ecc7bdf6-9284-4979-a1a6-86cfcd379eb2_700x449.jpg}
\caption{Lage der Vibe Coding}
\end{figure}  
Außerdem wen wir von Lernen reden , reden wir auch von Fehler machen, Geduld und dabei der Prozess des Lernens 
zu genießen. Vibe Coding kann diesen Prozess stören, da es den Lernenden ermöglicht, einen Schnelleren Lösung
zu finden, ohne die zugrunde liegenden Konzepte zu verstehen. Wenn wir sagen wir Sind Softwareentwickler, Student in 
der Informatik, sagen wir auch, dass wir Geduld haben wollen, dass wir bereit sind, Fehler zu machen und daraus zu lernen.
aber Vibe Coding ist nur da um dieses wunderbaren Geschenke um uns zu nehmen. 

\begin{quote}
A Meister in Software development was a student, but a Vibe Coder was never a student, will never be student or Meister, but will remain forever a Vibe Coder, who will never know what an Abstract method is, what a polymorphism is.
\end{quote}
\citep{Unknown}

\section{Fazit und Schlussfolgerungen}
Vibe Coder fühlt sich als Softwareentwickler, aber wird Softwareentwickler.
Vibe Coding kann gut sein , weil man schnelleren Lösung bekommt aber das effiziente und Gute Lösung
zu bekommen ist durch das ware Einsatz der Softwareentwicklung, wo Dokumentation liest, wo Geduld und Fehler machen,
wo man mit Menschen interagieren um einen Lösung zu bekommen, wo man in verschiedene Lernquelle nach Lösung sucht. 



\subsection{Offene Fragen}
Wird in Zukunft so sein dass wir mehr in Bereich der Softwareentwicklung auf KI-Tools abhängig sein werden ?
Wird dadurch die Weiterentwicklung von Menschen, die im Bereich der Softwareentwicklung arbeiten, beeinträchtigt?
Wie wird die Qualität von Software in Zukunft aussehen, wenn wir uns zu sehr auf KI-Tools verlassen?

\subsection{Diskussion}

% Hier beginnt das Literaturverzeichnis
\clearpage
\renewcommand\refname{Literaturverzeichnis}
\bibliographystyle{alpha}
\bibliography{literatur}
\addcontentsline{toc}{section}{Literaturverzeichnis}


% Hier beginnt der Anhang
\clearpage
\appendix
\part*{Anhang}
\addcontentsline{toc}{section}{Anhang}

\section{Datenmaterial}

Für die Analyse und Bewertung von Vibe Coding wurden verschiedene Quellen herangezogen. Eine zentrale Grundlage bildet das Buch \textit{Vibe Coding: The Future of Programming} von Addy Osmani \citep{VibeCodingTheFutureOfProgramming}. In diesem Werk werden aktuelle Entwicklungen, Anwendungsbeispiele und Herausforderungen im Bereich KI-gestützter Programmierung umfassend dargestellt.

Zusätzlich wurden weitere wissenschaftliche Artikel, Blogbeiträge und Erfahrungsberichte ausgewertet, um ein möglichst breites und aktuelles Datenmaterial zu erhalten. Die Kombination dieser Quellen ermöglicht eine fundierte Analyse der Potenziale und Grenzen von Vibe Coding in

\end{document}

